\documentclass[12pt]{article}
\usepackage[utf8]{inputenc}

\usepackage[margin=1in]{geometry}
\usepackage{lipsum}

\usepackage[backend=biber,style=ieee]{biblatex}
\addbibresource{sources.bib}

\usepackage{titling}
\newcommand{\subtitle}[1]{%
	\posttitle{%
		\par\end{center}
	\begin{center}\large#1\end{center}
	\vskip0.1em}}%

\title{Reflection\\
Sustainable Development}
\subtitle{PEGN 430A}
\author{Tyler Singleton}
\date{07 March 2022}

\begin{document}
\maketitle

\newpage
\setlength{\parindent}{0pt}

% --- Questions Section --- %
\textbf{Questions} \\

% Question 1
\textbf{1. List and describe the three elements of Sustainable Development.} \\

The three elements of sustainable development as described by Dr. Battalora:

\begin{enumerate}
    \item Social \\
    The social element was described as identifying issues which directly impact peoples quality of life. Examples of this can include accesses to clean drinking water, gender equality, and access to energy. 
    
    \item Economic \\
    Dr. Battalora presented the economic element to focus on distribution of limited resources to the people in-order to improve their lives. An example of this element include the distribution of grain and food among poverty or those experiencing hardships.
    
    \item Environment \\
    The environmental element looks at the natural resources (renewable and non-renewable) available within the local area. Examples of this include oil production, marine life (fish, marine mammals, and coral reefs), and timber for construction.
\end{enumerate}

% Question 2
\textbf{2.What are the United Nations Sustainability Development Goals (UN SDGs)?  What is their overall purpose?} \\

As presented by Dr. Battalora, the United Nations (UN) have 17 Sustainability Development Goals (SDG). These goals range from reducing poverty, access to affordable and clean energy, and peace between nations. The purpose of UN SDGs revolve around transforming our global current unsustainable practices into a sustainable framework that allows for social inclusion, environmental sustainability, and economic development for today's generation and tomorrows. \\

% Question 3
\textbf{3. Consider the types of projects or operations in your discipline (field of study).  Refer to the 17 UN SDGS.  Identify at least three SDGs that apply to one or more these projects.  Briefly describe the project or operations, and discuss why the SDGs apply.} \\

My field of study revolves around geophysical engineering. Deep-sea mining is a recent project that is associated within this field. This project involves obtaining precious metals and minerals deposited from hydro-thermal vents along the seafloor. Three UN SDGs that apply to his project are: \textit{(1)} Life Below Water, \textit{(2)} Work and Economic Growth, and \textit{(3)} Peace, Justice, and Strong Institutions. \\

Life Below Water applies because the mining operation for these minerals uplift sand into suspension. This can damage the local underwater ecosystem, and create areas that are uninhabitable for marine life. Work and Economic Growth applies because terrestrial harvesting of minerals is decaying. Deep-sea mining is an innovating field which will create decent jobs in engineering and technology as to develop these operations will require extensive studies on the local marine ecology, engineering the tools to operate a deep-sea mine, and jobs to refine and transport the minerals. Finally, Peace, Justice, and Strong Institutions applies because these operations may take place in international waters. Therefore, cooperation between nations to prevent harassment of these operations will be required. 

\end{document}
